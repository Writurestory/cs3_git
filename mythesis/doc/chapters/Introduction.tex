\chapter{前言}

\section{研究背景}

\begin{quote}
  ``提供WiFi服务每年我们每一家分店额外增加的客流量是15000。''
  \flushright{\emph{--- John Wooley, president and CEO of Schlotzsky s Deli (From
      Chainleader.com)}}
\end{quote}


\begin{quote}
  ``WiFi服务,有助于吸引顾客和提高非高峰期的客流量。这些客人往往是"高收入"人士,他们习惯经常
  光顾,并停留较长的时间。''
  \flushright{\emph{--- Lovina McMurchy, director}}
\end{quote}

在国外一些酒店、餐厅、咖啡厅、商业中心或其他的商家,我们经常可以看到可以使用笔记本或手机使
用商家提供的无线WiFi网络上网。这是一个非常好的手段,客人得到了便利,商家赢得了客流。任何咖
啡店、酒店等的成功经营,关键在于吸引新客户和留住现有客户。提供无线上网服务,可帮助你达到这
两个目标。很多商务人士、旅客和学生,即使他们在休假中,都往往需要跟外界保持各种各样的定期接
触和联络。越来越多的人,如销售人员和客户代表等等,倾向于离开办公室,在一个较轻松自在的环境
下,跟他们的客户会面。提供WiFi服务绝对给他们一个额外的理由光顾你的场所,并且留下来作更多的
消费。WiFi无线上网服务,可吸引更多的企业来到你的场所举行早餐或午餐聚会,同时也可吸引商务旅
客到此用膳或逗留这样,你的场所就可作为这些公司及其雇员一个重要聚脚点,让他们轻松聚会的同时
又可维持有一个高效率的工作环境。

目前国内的很多商业场所提供免费WIFI服务,例如酒店、咖啡店、餐饮店、机场等等,但是国内的这些
场所提供的部分WIFI热点认证存在认证繁琐、人工操作、不够安全稳定等问题。认证繁琐确实让人头疼,
就以北京的首都机场为例,它为旅客提供了以下几种获得WiFi上网账号的认证方式:在航站楼各处自助
一体机上获取,在首都机场WiFi主页输入国内手机号码通过短信获取,使用腾讯微博或QQ账号登录,中
国电信和中国联通手机用户从独立的认证窗口输入手机号码及事先获得的Wi-Fi上网密码。方式似乎有很
多,但了解起来却不是那么轻松,而且即使在正确无误地输入密码后,系统有时也反复提示密码错误,
几次失败的尝试足以破坏用户使用WiFi的兴致。

看起来使用WiFi的用户似乎格外挑剔。要知道,WiFi一直被视作一种灵活、带宽大、使用率高的技术,
“无缝覆盖”更是广为流传。因此,用户就自然对WiFi期望颇高:方便、便宜、快捷。提高WiFi的易用性,
显然是必须提前做好的准备功课。

\section{系统概述}

该系统集成了Raspberry Pi+刷卡器+打印机,使用一套PYTHON程序控制MikroTik ROS路由器实
现hotspot\cite{wifi-hotspot}认证功能,客户刷卡后打印机自动打印含有用户名密码的纸条,客户接
入网络后会弹出认证界面,输入纸条上的用户名密码上网。用户如果长时间不上网便会自动下线,时间
可以自己定义,下次还需要上网只需要再输入纸条上的用户名密码,有效利用了网络资源。比如入住酒
店的用户在入住的时候就可以使用房卡刷一次,取纸条上网,只要客户不退房,可以凭纸条一直上网,
当客户离开酒店的时候只需要再刷一次,自动下线纸条就作废了,酒店管理人员不用手工操作,非常方
便。

\section{系统特色}

经过了一定的市场调查和分析后,我们针对国内WiFi热点认证的方案的优缺点设计出了一套基
于Raspberry Pi的网络用户认证管理系统,主要有以下特色:

\begin{description}
\item[自动化] 基于Raspberry Pi集成的PYTHON程序实现自动认证功能,认证的方式无需工作人员手动
  操作,不需要手动添加上网帐号和密码,只需要把Raspberry Pi接入咖啡厅、酒店、或者机场的网络
  即可。
\item[刷卡打印认证] 顾客上网只需要刷卡,商家可以自定义用卡的种类来进行认证,比如说用购物卡、酒店房卡、公
  交卡、等等都可以用来认证,然后打印机自动打印纸条,凭借纸条上的用户名密码,客户接入热点信
  号自动弹出认证界面,输入纸条上给的用户名密码上网。
\item[配置部署] 该系统集成于Raspberry Pi,程序存储与SD卡内,部署的时候只需要将SD卡插入树莓
  派,ROS路由器接入商家的网络,修改一些参数即可成功部署,可以快速部署。
\item[稳定] 如果Raspberry Pi损坏了,不会影响网络的使用,只是暂时不能刷卡,只需跟换一个新的
  Raspberry Pi然后插入SD卡即可恢复。如果遇到断电或者人为把电拔了,恢复供电等10秒钟就又可以继
  续刷卡认证了。商家可以自定义卡的种类,因此使用其他卡是不能认证的,更安全,有效防止蹭网和
  恶意软件的攻击。
\item[有线认证] 比如酒店、学校等场所,不仅可以同时实现无线热点的认证,还可以实现有线网络的
  热点认证,该系统使用MikroTik ROS\cite{ros}作为认证路由器,不仅可以支持无线认证,同时也支持有线认证。
\item[WEB认证] WiFi热点认证的路的窗口可以作为商家的广告招牌,可以为商家自定义个性化广告界面。
\item[自动下线] 可以在ROS路由器内定义自动下线时间。如果用户想要自己注销,只需要再刷一次卡
  (下线打印机不出纸条)即可下线。
\item[安全] 输入方式只能通过刷卡,只有商家自己定义的卡才能刷,其他卡不能刷,有效防止了恶意软件
  消耗网络资源的行为。
\end{description}


%%% Local Variables: 
%%% mode: latex
%%% TeX-master: "../thesis"
%%% End: 
